%%  'printversion' pro sazbu verze pro tisk (nebarevné logo a odkazy,
%%  odkazy s uvedením adresy za odkazem, ne odkazy do rejstříku),
%%  jinak verze pro prohlížeč

%%  'biblatex' pro zapnutí podpory pro sazbu bibliografie pomocí
%%  BibLaTeXu, jinak je výchozí sazba v prostředí thebibliography

%%  'language=jazyk' pro jazyk práce, jazyky english pro anglický,
%%  slovak pro slovenský, jinak je výchozí czech pro český

%%  'font=sans' pro bezpatkový font (Iwona Light), jinak je výchozí
%%  serif pro patkový (Latin Modern)

%%  'figures, tables, theorems a sourcecodes' pro sazbu seznamu
%%  obrázků, tabulek, vět a zdrojových kódů, jinak při =false se
%%  nesází (u theorems a sourcecodes výchozí)

\documentclass[
%  printversion,
  biblatex,
  language=english,
%  font=sans,
  figures=false,
%  tables=false,
  sourcecodes,
  glossaries,
  index
]{kidiplom}

%% Informace pro úvodní strany. V jazyku práce (pokud není v komentáři
%% uvedeno česky) a anglicky. Uveďte všechny, u kterých není v
%% komentáři uvedeno, že jsou volitelné. Při neuvedení se použijí
%% výchozí texty. Text pro jiný než nastavený jazyk práce (nepovinným
%% parametrem language makra \documentclass, výchozí český) se zadává
%% použitím makra s uvedením jazyka jako nepovinného parametru.

%% Název práce, česky a anglicky. Měl by se vysázet na jeden řádek.
\title[czech]{Vytvoření komplexní aplikace pro iOS pomocí SwiftUI a Swift Backend Development}
\title[english]{Building a comprehensive iOS application with SwiftUI and Swift Backend Development}

%% Jméno autora práce. Makro nemá nepovinný parametr pro uvedení
%% jazyka.
\author{Maksym Kupchenko}

%% Jméno vedoucího práce (včetně titulů). Makro nemá nepovinný
%% parametr pro uvedení jazyka.
\supervisor{Mgr. Roman Vyjídáček, Ph.D.}

%% Volitelný rok odevzdání práce. Výchozí je aktuální (kalendářní)
%% rok. Makro nemá nepovinný parametr pro uvedení jazyka.
\yearofsubmit{2024}

%% Anotace práce, včetně anglické (obvykle překlad z jazyka
%% práce). Jeden odstavec!
\annotation[czech]{Závěrečná práce se zabývá vývojem serveru Vapor s PostgreSQL a mobilní aplikací SwiftUI pro iOS. Zahrnuje vývoj Swift na straně serveru, ověřování, ověřování dat a zpracování chyb. Práce hodnotí výkon, použitelnost a bezpečnost. Přispívá ke znalostem o Swift na straně serveru a ukazuje potenciál Vapor a SwiftUI pro škálovatelná řešení aplikací pro iOS.}

\annotation[english]{Bachelor thesis explores the development of a Vapor server with PostgreSQL and a SwiftUI iOS application. It covers server-side Swift development, authentication, data validation, and error handling. The work evaluates performance, usability, and security. It contributes to knowledge on server-side Swift, showcasing the potential of Vapor and SwiftUI for scalable iOS app solutions.}

%% Klíčová slova práce, včetně anglických. Oddělená (obvykle) středníkem.
\keywords[czech]{Swift, Vapor, SwiftUI, iOS, PostgreSQL}
\keywords[english]{Swift, Vapor, SwiftUI, iOS, PostgreSQL}

%% Volitelná specifikace příloh textu práce, i anglicky. Výchozí je
%% 'elektronická data v systému katedry informatiky / electronic data
%% in system of department of computer science'.
%\supplements{nejlepší software všech dob}
%\supplements[english]{the best software of all times}

%% Volitelné poděkování. Stručné! Výchozí je prázdné. Makro nemá
%% nepovinný parametr pro uvedení jazyka.

%% Cesta k souboru s bibliografií pro její sazbu pomocí BibLaTeXu
%% (zvolenou nepovinným parametrem biblatex makra
%% \documentclass). Použijte pouze při této sazbě, ne při (výchozí)
%% sazbě v prostředí thebibliography.
\bibliography{bibliografie.bib}

%% Další dodatečné styly (balíky) potřebné pro sazbu vlastního textu
%% práce.
\usepackage{lipsum}
\usepackage{longtable}

\begin{document}
%% Sazba úvodních stran -- titulní, s bibliografickými údaji, s
%% anotací a klíčovými slovy, s poděkováním a prohlášením, s obsahem a
%% se seznamy obrázků, tabulek, vět a zdrojových kódů (pokud jejich
%% sazba není vypnutá).
\maketitle

%% Vlastní text závěrečné práce. Pro povinné závěry, před přílohami,
%% použijte prostředí kiconclusions. Povinná je i příloha s obsahem
%% elektronických dat.

%% -------------------------------------------------------------------

\newcommand{\BibLaTeX}{\textsc{Bib}\LaTeX}

\section{Introduction}

\subsection{Background and motivation for the project}
\subsection{Research questions and objectives}
\subsection{Scope and limitations of the project}

\section{Literature Review}
\subsection{Overview of server-side Swift development with Vapor}
\subsection{PostgreSQL as a database solution}
\subsection{Review of SwiftUI for iOS app development}

\section{Server-side Development}
\subsection{Exploring Vapor framework and its key components}
\subsection{Building RESTful APIs with Vapor and PostgreSQL}
\subsection{Authentication and authorization mechanisms}
\subsection{Handling data validation and error handling on the server side}

\section{Client-side Development}
\subsection{Introduction to SwiftUI and its core concepts}
\subsection{Designing the user interface using SwiftUI components}
\subsection{Consuming RESTful APIs from the Vapor server}

\section{Integration and Testing}
\subsection{Integrating the Vapor server with the SwiftUI application}
\subsection{Performing unit testing for the server}
\subsection{UI testing for mobile app}


\begin{kiconclusions}
Conclusions
\end{kiconclusions}

%% Přílohy obsahu textu práce, za makrem \appendix.
\appendix

\section{Appendix}

\section{Sources} \label{sec:ObsahData}

%% Sazba volitelného rejstříku, za bibliografií.
\printindex

\end{document}

